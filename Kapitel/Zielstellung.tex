% !TEX root =../main.tex

\chapter{Zielstellung}

%Beschreiben Sie Ihre konkreten Aufgaben und Tätigkeiten während des Praktikums. Gehen Sie dabei auch auf notwendige und hilfreiche Kompetenzen und persönliche Voraussetzungen ein.

%Neugestaltung der Webseite mit lewishowes.com als Vorbild.

Die Haupt-Webseite der \term{HERO SOCIETY} ist \url{http://hero-society.org}. Darüber hinaus existieren weitere Webseiten für einzelne Coaches und Trainer der \term{HERO SOCIETY}. Eine dieser zusätzlichen Webseiten ist \url{http://mitchsenf.com} und sie soll im Rahmen des hier beschriebenen Praktikums überarbeitet werden.

Als Vorlage für die Überarbeitung soll eine bereits existierende Webseite dienen. Der Praxispartner hat hierfür die Webseite \url{https://lewishowes.com/} ausgewählt. Beide Webseiten beschäftigen sich mit dem Thema Karriereberatung und Coaching und haben laut Praxispartner ähnliche Inhalte und Zielgruppen. Da sowohl \url{http://mitchsenf.com} als auch \url{https://lewishowes.com/} mit Hilfe von WordPress erstellt wurden, besteht die Annahme, dass das Design der genannten Webseiten angeglichen werden kann.

Folgende Ziele wurden für das Praktikum in Zusammenarbeit mit dem Praxispartner definiert:

\begin{enumerate}
	\item Untersuchen, welches WordPress-Theme für die Webseite \url{https://lewishowes.com/} eingesetzt wird
	\item Untersuchen, ob bestehendes WordPress-Theme von \url{http://mitchsenf.com} angepasst werden kann, oder ob ein neues WordPress-Theme ausgewählt werden muss
	\item Entwicklungssystem aufsetzen, um Entwicklungsarbeit unabhängig vom Produktiv-System durchführen zu können
	\item Frontendlogik für Darstellung und Animationen in \url{https://lewishowes.com/} untersuchen und gegebenenfalls für \url{http://mitchsenf.com} reimplementieren
	\item Backendlogik für das registrieren für Newsletter implementieren
	\item Angepasstes Design muss auf verschiedenen Display-Größen wie Monitor, Tablet und Handy korrekt dargestellt werden
	\item Möglichkeit schaffen, um die Webseite \url{http://mitchsenf.com} vom bisherigen Design zum neuen Design zu migrieren
\end{enumerate}
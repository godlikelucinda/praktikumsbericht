% !TEX root =../main.tex

\chapter{Reflexion}

In diesem Kapitel wird das Vorgehen, die erzielten Ergebnisse, sowie die gesammelten Erfahrungen zusammengefasst.


\section{Erwartungen und Ziele}

%Welche Erwartungen hatten Sie an Ihr Praktikum und welche Ziele haben Sie mit diesem Praktikum verfolgt? In welchem Umfang wurden die Ziele erreicht? Falls Abstriche, warum?

Neben den, in Zusammenarbeit mit dem Praxispartner, definierten Ziele, wurden durch die Autorin folgende Erwartungen und Ziele an das Praktikum gestellt:

\begin{itemize}
	\item Kenntnisse im Bereich WordPress vertiefen
	\item Praxiserfahrung im Bereich Web-Development erlangen
	\item JavaScript anwenden
	\item PHP anwenden
\end{itemize}

Die Erwartungen der Autorin an das Praktikum und den Praktikumspartner wurden erfüllt. Die gewünschten Erfahrungen konnten erlangt und die gewünschten Programmiersprachen angewendet. Es konnten sowohl Lösungen in JavaScript als auch in PHP implementiert werden.


\section{Aufgaben und Tätigkeiten}

%Bitte beschreiben Sie ausführlich Ihre Aufgaben und Tätigkeiten während des Praktikums (Einarbeitung und Betreuung, Aufgabenstellung, Bearbeitung/Ausführung, Ergebnis)

Im Rahmen des Praktikums wurde die Webseite \url{https://lewishowes.com/} untersucht. Besonderes Augenmerk lag hierbei auf der Untersuchung des verwendeten WordPress-Themes.

Die Webseite \url{https://lewishowes.com/} verwendet ein auf dem Genesis-Framework\footcite{genesis-theme} basierendes Theme. Dieses Theme wurde speziell für \url{https://lewishowes.com/} erstellt und steht nicht in einem Theme-Store zur Verfügung. Es konnte daher nicht direkt für die Webseite \url{http://mitchsenf.com} verwendet werden. Stattdessen wurde im Ersten Schritt versucht, das bestehende Theme von \url{http://mitchsenf.com} zu modifizieren.

Das von \url{http://mitchsenf.com} eingesetzte Theme trägt des Titel \term{Story}. Im Rahmen des Praktikums wurde versucht, das Theme um benötigte Funktionalität zu erweitern. Beispielsweise wurde versucht einen eigenen Fullscreenslider zu erstellen, welcher sich automatisch der Displaygröße anpasst.

Das Erweitern des Story-Themes hat sich als nicht praktikabel erwiesen. Es existieren Kompatibilitätsprobleme zwischen Story-Theme und jQuery. Somit konnte jQuery für die Erweiterung des Themes nicht verwendet werden. Im Laufe des Praktikums wurde der Aufwand als zu hoch eingeschätzt und die Suche nach einem alternativen Theme wurde begonnen.

Als zu verwendendes Theme wurde \term{Sydney} ausgewählt. Dieses Theme steht in einer Basis-Variante kostenfrei zur Verfügung. Zusätzlich wurde das Page-Builder Plugin \term{Elementor} verwendet. Mit Hilfe von \term{Elementor} konnte eigener JavaScript-Code komfortabel und Fehlerfrei eingefügt werden.

Für einen Großteil der einzufügenden Inhalte wurde der Page-Builder \term{Elementor} verwendet. Alle weiteren Funktionalitäten wurden im Rahmen des Praktikums per PHP oder JavaScript implementiert.


\section{Werkzeuge und Methoden}

%Welche Werkzeuge und Methoden wurden eingesetzt?

Folgende Werkzeuge wurden im Rahmen des Praktikums eingesetzt:

\begin{itemize}
	\item WordPress\footcite{wordpress-homepage}
	\item Theme Sydney\footcite{sydney-homepage}
	\item Page-Builder Elementor\footcite{elementor-homepage}
	\item PHP\footcite{php-homepage}
	\item JavaScript\footcite{javascript-homepage}
	\item Texteditor Atom\footcite{atom-homepage}
\end{itemize}


\section{Kenntnisse und Fähigkeiten}

%Welche Kenntnisse und Fähigkeiten (Fach-, Sozial-, Methodenkompetenzen) konnten Sie durch Ihr Praktikum erwerben oder vertiefen?

Das Praktikum wurde mit wöchentlichen Reviews durchgeführt, sodass Kenntnisse im Bereich der agilen Softwareentwicklung gesammelt werden können. Auch Kenntnisse im Bereich der Kundenkommunikation und Projektplanung konnten gesammelt werden. Es musste abgewogen werden, welche Änderungswünsche innerhalb des Praktikums umgesetzt werden können und wie dies dem Praktikumspartner zu kommunizieren ist.

Aus technischer Sicht konnten Kenntnisse im Bereich der Webentwicklung im Allgemeinen und WordPress im speziellen gesammelt werden. Darüber hinaus konnten praktische Erfahrungen im Einsatz von PHP und JavaScript gesammelt werden.


\section{Zusammenhang von Studium und Praktikum}

%Welcher Zusammenhang bestand zwischen Studium und Praktikum, bzw. welchen Nutzen ziehen Sie aus Ihrem Praktikum?

Studium und Praktikum haben vor allen in den Bereichen Programmierung und Projektmanagement große Schnittmengen. Das im Studium erlangte wissen konnte im Rahmen des Praktikums eingesetzt und zum Teil vertieft werden.

Der Nutzen des Praktikums liegt für die Autorin darin, das erworbene Wissen in einem realen Projekt einsetzen zu können und ein reales Problem zu lösen.


\section{Vorkenntnisse}

%Inwieweit konnten behandelte Studieninhalte angewendet werden? Welche Voraussetzungen fehlten? Gibt es individuelle Hinweise, wie Studieninhalte geändert werden könnten, um besser auf das Praktikum vorzubereiten?

Folgende bereits im Studium behandelte Inhalte konnten im Rahmen des Praktikums angewendet werden.

\begin{itemize}
	\item HTML
	\item CSS
	\item jQuery
	\item SQL
	\item Relationale Datenbanken
\end{itemize}


\section{Bewertung}

%Würden Sie anderen Studierenden empfehlen, in diesem Unternehmen ein Praktikum zu absolvieren (mit Begründung)?

Die Zusammenarbeit mit dem Praxispartner und mit dem Betreuer Michael Senf war stets konstruktiv. Im Rahmen des Praktikums konnte eine interessante Aufgabe bearbeitet und Kenntnisse in den Bereichen Webentwicklung, WordPress, WordPress-Themes, WordPress-Plugins, WordPress-Page-Builder, PHP, JavaScript, HTML und CSS gesammelt und vertieft werden.

Besonders hervorzuheben ist hierbei die persönliche Betreuung durch Michael Senf, der stets mit Engagement dafür gesorgt hat, das Projekt weiter voranzubringen ohne Termindruck aufkommen zu lassen.

Ein Praktikum bei der \term{HERO SOCIETY} ist daher laut Meinung der Autorin uneingeschränkt zu empfehlen.
